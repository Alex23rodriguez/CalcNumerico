
% Default to the notebook output style

    


% Inherit from the specified cell style.




    
\documentclass[11pt]{article}

    
    
    \usepackage[T1]{fontenc}
    % Nicer default font (+ math font) than Computer Modern for most use cases
    \usepackage{mathpazo}

    % Basic figure setup, for now with no caption control since it's done
    % automatically by Pandoc (which extracts ![](path) syntax from Markdown).
    \usepackage{graphicx}
    % We will generate all images so they have a width \maxwidth. This means
    % that they will get their normal width if they fit onto the page, but
    % are scaled down if they would overflow the margins.
    \makeatletter
    \def\maxwidth{\ifdim\Gin@nat@width>\linewidth\linewidth
    \else\Gin@nat@width\fi}
    \makeatother
    \let\Oldincludegraphics\includegraphics
    % Set max figure width to be 80% of text width, for now hardcoded.
    \renewcommand{\includegraphics}[1]{\Oldincludegraphics[width=.8\maxwidth]{#1}}
    % Ensure that by default, figures have no caption (until we provide a
    % proper Figure object with a Caption API and a way to capture that
    % in the conversion process - todo).
    \usepackage{caption}
    \DeclareCaptionLabelFormat{nolabel}{}
    \captionsetup{labelformat=nolabel}

    \usepackage{adjustbox} % Used to constrain images to a maximum size 
    \usepackage{xcolor} % Allow colors to be defined
    \usepackage{enumerate} % Needed for markdown enumerations to work
    \usepackage{geometry} % Used to adjust the document margins
    \usepackage{amsmath} % Equations
    \usepackage{amssymb} % Equations
    \usepackage{textcomp} % defines textquotesingle
    % Hack from http://tex.stackexchange.com/a/47451/13684:
    \AtBeginDocument{%
        \def\PYZsq{\textquotesingle}% Upright quotes in Pygmentized code
    }
    \usepackage{upquote} % Upright quotes for verbatim code
    \usepackage{eurosym} % defines \euro
    \usepackage[mathletters]{ucs} % Extended unicode (utf-8) support
    \usepackage[utf8x]{inputenc} % Allow utf-8 characters in the tex document
    \usepackage{fancyvrb} % verbatim replacement that allows latex
    \usepackage{grffile} % extends the file name processing of package graphics 
                         % to support a larger range 
    % The hyperref package gives us a pdf with properly built
    % internal navigation ('pdf bookmarks' for the table of contents,
    % internal cross-reference links, web links for URLs, etc.)
    \usepackage{hyperref}
    \usepackage{longtable} % longtable support required by pandoc >1.10
    \usepackage{booktabs}  % table support for pandoc > 1.12.2
    \usepackage[inline]{enumitem} % IRkernel/repr support (it uses the enumerate* environment)
    \usepackage[normalem]{ulem} % ulem is needed to support strikethroughs (\sout)
                                % normalem makes italics be italics, not underlines
    

    
    
    % Colors for the hyperref package
    \definecolor{urlcolor}{rgb}{0,.145,.698}
    \definecolor{linkcolor}{rgb}{.71,0.21,0.01}
    \definecolor{citecolor}{rgb}{.12,.54,.11}

    % ANSI colors
    \definecolor{ansi-black}{HTML}{3E424D}
    \definecolor{ansi-black-intense}{HTML}{282C36}
    \definecolor{ansi-red}{HTML}{E75C58}
    \definecolor{ansi-red-intense}{HTML}{B22B31}
    \definecolor{ansi-green}{HTML}{00A250}
    \definecolor{ansi-green-intense}{HTML}{007427}
    \definecolor{ansi-yellow}{HTML}{DDB62B}
    \definecolor{ansi-yellow-intense}{HTML}{B27D12}
    \definecolor{ansi-blue}{HTML}{208FFB}
    \definecolor{ansi-blue-intense}{HTML}{0065CA}
    \definecolor{ansi-magenta}{HTML}{D160C4}
    \definecolor{ansi-magenta-intense}{HTML}{A03196}
    \definecolor{ansi-cyan}{HTML}{60C6C8}
    \definecolor{ansi-cyan-intense}{HTML}{258F8F}
    \definecolor{ansi-white}{HTML}{C5C1B4}
    \definecolor{ansi-white-intense}{HTML}{A1A6B2}

    % commands and environments needed by pandoc snippets
    % extracted from the output of `pandoc -s`
    \providecommand{\tightlist}{%
      \setlength{\itemsep}{0pt}\setlength{\parskip}{0pt}}
    \DefineVerbatimEnvironment{Highlighting}{Verbatim}{commandchars=\\\{\}}
    % Add ',fontsize=\small' for more characters per line
    \newenvironment{Shaded}{}{}
    \newcommand{\KeywordTok}[1]{\textcolor[rgb]{0.00,0.44,0.13}{\textbf{{#1}}}}
    \newcommand{\DataTypeTok}[1]{\textcolor[rgb]{0.56,0.13,0.00}{{#1}}}
    \newcommand{\DecValTok}[1]{\textcolor[rgb]{0.25,0.63,0.44}{{#1}}}
    \newcommand{\BaseNTok}[1]{\textcolor[rgb]{0.25,0.63,0.44}{{#1}}}
    \newcommand{\FloatTok}[1]{\textcolor[rgb]{0.25,0.63,0.44}{{#1}}}
    \newcommand{\CharTok}[1]{\textcolor[rgb]{0.25,0.44,0.63}{{#1}}}
    \newcommand{\StringTok}[1]{\textcolor[rgb]{0.25,0.44,0.63}{{#1}}}
    \newcommand{\CommentTok}[1]{\textcolor[rgb]{0.38,0.63,0.69}{\textit{{#1}}}}
    \newcommand{\OtherTok}[1]{\textcolor[rgb]{0.00,0.44,0.13}{{#1}}}
    \newcommand{\AlertTok}[1]{\textcolor[rgb]{1.00,0.00,0.00}{\textbf{{#1}}}}
    \newcommand{\FunctionTok}[1]{\textcolor[rgb]{0.02,0.16,0.49}{{#1}}}
    \newcommand{\RegionMarkerTok}[1]{{#1}}
    \newcommand{\ErrorTok}[1]{\textcolor[rgb]{1.00,0.00,0.00}{\textbf{{#1}}}}
    \newcommand{\NormalTok}[1]{{#1}}
    
    % Additional commands for more recent versions of Pandoc
    \newcommand{\ConstantTok}[1]{\textcolor[rgb]{0.53,0.00,0.00}{{#1}}}
    \newcommand{\SpecialCharTok}[1]{\textcolor[rgb]{0.25,0.44,0.63}{{#1}}}
    \newcommand{\VerbatimStringTok}[1]{\textcolor[rgb]{0.25,0.44,0.63}{{#1}}}
    \newcommand{\SpecialStringTok}[1]{\textcolor[rgb]{0.73,0.40,0.53}{{#1}}}
    \newcommand{\ImportTok}[1]{{#1}}
    \newcommand{\DocumentationTok}[1]{\textcolor[rgb]{0.73,0.13,0.13}{\textit{{#1}}}}
    \newcommand{\AnnotationTok}[1]{\textcolor[rgb]{0.38,0.63,0.69}{\textbf{\textit{{#1}}}}}
    \newcommand{\CommentVarTok}[1]{\textcolor[rgb]{0.38,0.63,0.69}{\textbf{\textit{{#1}}}}}
    \newcommand{\VariableTok}[1]{\textcolor[rgb]{0.10,0.09,0.49}{{#1}}}
    \newcommand{\ControlFlowTok}[1]{\textcolor[rgb]{0.00,0.44,0.13}{\textbf{{#1}}}}
    \newcommand{\OperatorTok}[1]{\textcolor[rgb]{0.40,0.40,0.40}{{#1}}}
    \newcommand{\BuiltInTok}[1]{{#1}}
    \newcommand{\ExtensionTok}[1]{{#1}}
    \newcommand{\PreprocessorTok}[1]{\textcolor[rgb]{0.74,0.48,0.00}{{#1}}}
    \newcommand{\AttributeTok}[1]{\textcolor[rgb]{0.49,0.56,0.16}{{#1}}}
    \newcommand{\InformationTok}[1]{\textcolor[rgb]{0.38,0.63,0.69}{\textbf{\textit{{#1}}}}}
    \newcommand{\WarningTok}[1]{\textcolor[rgb]{0.38,0.63,0.69}{\textbf{\textit{{#1}}}}}
    
    
    % Define a nice break command that doesn't care if a line doesn't already
    % exist.
    \def\br{\hspace*{\fill} \\* }
    % Math Jax compatability definitions
    \def\gt{>}
    \def\lt{<}
    % Document parameters
    
    
    
    

    % Pygments definitions
    
\makeatletter
\def\PY@reset{\let\PY@it=\relax \let\PY@bf=\relax%
    \let\PY@ul=\relax \let\PY@tc=\relax%
    \let\PY@bc=\relax \let\PY@ff=\relax}
\def\PY@tok#1{\csname PY@tok@#1\endcsname}
\def\PY@toks#1+{\ifx\relax#1\empty\else%
    \PY@tok{#1}\expandafter\PY@toks\fi}
\def\PY@do#1{\PY@bc{\PY@tc{\PY@ul{%
    \PY@it{\PY@bf{\PY@ff{#1}}}}}}}
\def\PY#1#2{\PY@reset\PY@toks#1+\relax+\PY@do{#2}}

\expandafter\def\csname PY@tok@w\endcsname{\def\PY@tc##1{\textcolor[rgb]{0.73,0.73,0.73}{##1}}}
\expandafter\def\csname PY@tok@c\endcsname{\let\PY@it=\textit\def\PY@tc##1{\textcolor[rgb]{0.25,0.50,0.50}{##1}}}
\expandafter\def\csname PY@tok@cp\endcsname{\def\PY@tc##1{\textcolor[rgb]{0.74,0.48,0.00}{##1}}}
\expandafter\def\csname PY@tok@k\endcsname{\let\PY@bf=\textbf\def\PY@tc##1{\textcolor[rgb]{0.00,0.50,0.00}{##1}}}
\expandafter\def\csname PY@tok@kp\endcsname{\def\PY@tc##1{\textcolor[rgb]{0.00,0.50,0.00}{##1}}}
\expandafter\def\csname PY@tok@kt\endcsname{\def\PY@tc##1{\textcolor[rgb]{0.69,0.00,0.25}{##1}}}
\expandafter\def\csname PY@tok@o\endcsname{\def\PY@tc##1{\textcolor[rgb]{0.40,0.40,0.40}{##1}}}
\expandafter\def\csname PY@tok@ow\endcsname{\let\PY@bf=\textbf\def\PY@tc##1{\textcolor[rgb]{0.67,0.13,1.00}{##1}}}
\expandafter\def\csname PY@tok@nb\endcsname{\def\PY@tc##1{\textcolor[rgb]{0.00,0.50,0.00}{##1}}}
\expandafter\def\csname PY@tok@nf\endcsname{\def\PY@tc##1{\textcolor[rgb]{0.00,0.00,1.00}{##1}}}
\expandafter\def\csname PY@tok@nc\endcsname{\let\PY@bf=\textbf\def\PY@tc##1{\textcolor[rgb]{0.00,0.00,1.00}{##1}}}
\expandafter\def\csname PY@tok@nn\endcsname{\let\PY@bf=\textbf\def\PY@tc##1{\textcolor[rgb]{0.00,0.00,1.00}{##1}}}
\expandafter\def\csname PY@tok@ne\endcsname{\let\PY@bf=\textbf\def\PY@tc##1{\textcolor[rgb]{0.82,0.25,0.23}{##1}}}
\expandafter\def\csname PY@tok@nv\endcsname{\def\PY@tc##1{\textcolor[rgb]{0.10,0.09,0.49}{##1}}}
\expandafter\def\csname PY@tok@no\endcsname{\def\PY@tc##1{\textcolor[rgb]{0.53,0.00,0.00}{##1}}}
\expandafter\def\csname PY@tok@nl\endcsname{\def\PY@tc##1{\textcolor[rgb]{0.63,0.63,0.00}{##1}}}
\expandafter\def\csname PY@tok@ni\endcsname{\let\PY@bf=\textbf\def\PY@tc##1{\textcolor[rgb]{0.60,0.60,0.60}{##1}}}
\expandafter\def\csname PY@tok@na\endcsname{\def\PY@tc##1{\textcolor[rgb]{0.49,0.56,0.16}{##1}}}
\expandafter\def\csname PY@tok@nt\endcsname{\let\PY@bf=\textbf\def\PY@tc##1{\textcolor[rgb]{0.00,0.50,0.00}{##1}}}
\expandafter\def\csname PY@tok@nd\endcsname{\def\PY@tc##1{\textcolor[rgb]{0.67,0.13,1.00}{##1}}}
\expandafter\def\csname PY@tok@s\endcsname{\def\PY@tc##1{\textcolor[rgb]{0.73,0.13,0.13}{##1}}}
\expandafter\def\csname PY@tok@sd\endcsname{\let\PY@it=\textit\def\PY@tc##1{\textcolor[rgb]{0.73,0.13,0.13}{##1}}}
\expandafter\def\csname PY@tok@si\endcsname{\let\PY@bf=\textbf\def\PY@tc##1{\textcolor[rgb]{0.73,0.40,0.53}{##1}}}
\expandafter\def\csname PY@tok@se\endcsname{\let\PY@bf=\textbf\def\PY@tc##1{\textcolor[rgb]{0.73,0.40,0.13}{##1}}}
\expandafter\def\csname PY@tok@sr\endcsname{\def\PY@tc##1{\textcolor[rgb]{0.73,0.40,0.53}{##1}}}
\expandafter\def\csname PY@tok@ss\endcsname{\def\PY@tc##1{\textcolor[rgb]{0.10,0.09,0.49}{##1}}}
\expandafter\def\csname PY@tok@sx\endcsname{\def\PY@tc##1{\textcolor[rgb]{0.00,0.50,0.00}{##1}}}
\expandafter\def\csname PY@tok@m\endcsname{\def\PY@tc##1{\textcolor[rgb]{0.40,0.40,0.40}{##1}}}
\expandafter\def\csname PY@tok@gh\endcsname{\let\PY@bf=\textbf\def\PY@tc##1{\textcolor[rgb]{0.00,0.00,0.50}{##1}}}
\expandafter\def\csname PY@tok@gu\endcsname{\let\PY@bf=\textbf\def\PY@tc##1{\textcolor[rgb]{0.50,0.00,0.50}{##1}}}
\expandafter\def\csname PY@tok@gd\endcsname{\def\PY@tc##1{\textcolor[rgb]{0.63,0.00,0.00}{##1}}}
\expandafter\def\csname PY@tok@gi\endcsname{\def\PY@tc##1{\textcolor[rgb]{0.00,0.63,0.00}{##1}}}
\expandafter\def\csname PY@tok@gr\endcsname{\def\PY@tc##1{\textcolor[rgb]{1.00,0.00,0.00}{##1}}}
\expandafter\def\csname PY@tok@ge\endcsname{\let\PY@it=\textit}
\expandafter\def\csname PY@tok@gs\endcsname{\let\PY@bf=\textbf}
\expandafter\def\csname PY@tok@gp\endcsname{\let\PY@bf=\textbf\def\PY@tc##1{\textcolor[rgb]{0.00,0.00,0.50}{##1}}}
\expandafter\def\csname PY@tok@go\endcsname{\def\PY@tc##1{\textcolor[rgb]{0.53,0.53,0.53}{##1}}}
\expandafter\def\csname PY@tok@gt\endcsname{\def\PY@tc##1{\textcolor[rgb]{0.00,0.27,0.87}{##1}}}
\expandafter\def\csname PY@tok@err\endcsname{\def\PY@bc##1{\setlength{\fboxsep}{0pt}\fcolorbox[rgb]{1.00,0.00,0.00}{1,1,1}{\strut ##1}}}
\expandafter\def\csname PY@tok@kc\endcsname{\let\PY@bf=\textbf\def\PY@tc##1{\textcolor[rgb]{0.00,0.50,0.00}{##1}}}
\expandafter\def\csname PY@tok@kd\endcsname{\let\PY@bf=\textbf\def\PY@tc##1{\textcolor[rgb]{0.00,0.50,0.00}{##1}}}
\expandafter\def\csname PY@tok@kn\endcsname{\let\PY@bf=\textbf\def\PY@tc##1{\textcolor[rgb]{0.00,0.50,0.00}{##1}}}
\expandafter\def\csname PY@tok@kr\endcsname{\let\PY@bf=\textbf\def\PY@tc##1{\textcolor[rgb]{0.00,0.50,0.00}{##1}}}
\expandafter\def\csname PY@tok@bp\endcsname{\def\PY@tc##1{\textcolor[rgb]{0.00,0.50,0.00}{##1}}}
\expandafter\def\csname PY@tok@fm\endcsname{\def\PY@tc##1{\textcolor[rgb]{0.00,0.00,1.00}{##1}}}
\expandafter\def\csname PY@tok@vc\endcsname{\def\PY@tc##1{\textcolor[rgb]{0.10,0.09,0.49}{##1}}}
\expandafter\def\csname PY@tok@vg\endcsname{\def\PY@tc##1{\textcolor[rgb]{0.10,0.09,0.49}{##1}}}
\expandafter\def\csname PY@tok@vi\endcsname{\def\PY@tc##1{\textcolor[rgb]{0.10,0.09,0.49}{##1}}}
\expandafter\def\csname PY@tok@vm\endcsname{\def\PY@tc##1{\textcolor[rgb]{0.10,0.09,0.49}{##1}}}
\expandafter\def\csname PY@tok@sa\endcsname{\def\PY@tc##1{\textcolor[rgb]{0.73,0.13,0.13}{##1}}}
\expandafter\def\csname PY@tok@sb\endcsname{\def\PY@tc##1{\textcolor[rgb]{0.73,0.13,0.13}{##1}}}
\expandafter\def\csname PY@tok@sc\endcsname{\def\PY@tc##1{\textcolor[rgb]{0.73,0.13,0.13}{##1}}}
\expandafter\def\csname PY@tok@dl\endcsname{\def\PY@tc##1{\textcolor[rgb]{0.73,0.13,0.13}{##1}}}
\expandafter\def\csname PY@tok@s2\endcsname{\def\PY@tc##1{\textcolor[rgb]{0.73,0.13,0.13}{##1}}}
\expandafter\def\csname PY@tok@sh\endcsname{\def\PY@tc##1{\textcolor[rgb]{0.73,0.13,0.13}{##1}}}
\expandafter\def\csname PY@tok@s1\endcsname{\def\PY@tc##1{\textcolor[rgb]{0.73,0.13,0.13}{##1}}}
\expandafter\def\csname PY@tok@mb\endcsname{\def\PY@tc##1{\textcolor[rgb]{0.40,0.40,0.40}{##1}}}
\expandafter\def\csname PY@tok@mf\endcsname{\def\PY@tc##1{\textcolor[rgb]{0.40,0.40,0.40}{##1}}}
\expandafter\def\csname PY@tok@mh\endcsname{\def\PY@tc##1{\textcolor[rgb]{0.40,0.40,0.40}{##1}}}
\expandafter\def\csname PY@tok@mi\endcsname{\def\PY@tc##1{\textcolor[rgb]{0.40,0.40,0.40}{##1}}}
\expandafter\def\csname PY@tok@il\endcsname{\def\PY@tc##1{\textcolor[rgb]{0.40,0.40,0.40}{##1}}}
\expandafter\def\csname PY@tok@mo\endcsname{\def\PY@tc##1{\textcolor[rgb]{0.40,0.40,0.40}{##1}}}
\expandafter\def\csname PY@tok@ch\endcsname{\let\PY@it=\textit\def\PY@tc##1{\textcolor[rgb]{0.25,0.50,0.50}{##1}}}
\expandafter\def\csname PY@tok@cm\endcsname{\let\PY@it=\textit\def\PY@tc##1{\textcolor[rgb]{0.25,0.50,0.50}{##1}}}
\expandafter\def\csname PY@tok@cpf\endcsname{\let\PY@it=\textit\def\PY@tc##1{\textcolor[rgb]{0.25,0.50,0.50}{##1}}}
\expandafter\def\csname PY@tok@c1\endcsname{\let\PY@it=\textit\def\PY@tc##1{\textcolor[rgb]{0.25,0.50,0.50}{##1}}}
\expandafter\def\csname PY@tok@cs\endcsname{\let\PY@it=\textit\def\PY@tc##1{\textcolor[rgb]{0.25,0.50,0.50}{##1}}}

\def\PYZbs{\char`\\}
\def\PYZus{\char`\_}
\def\PYZob{\char`\{}
\def\PYZcb{\char`\}}
\def\PYZca{\char`\^}
\def\PYZam{\char`\&}
\def\PYZlt{\char`\<}
\def\PYZgt{\char`\>}
\def\PYZsh{\char`\#}
\def\PYZpc{\char`\%}
\def\PYZdl{\char`\$}
\def\PYZhy{\char`\-}
\def\PYZsq{\char`\'}
\def\PYZdq{\char`\"}
\def\PYZti{\char`\~}
% for compatibility with earlier versions
\def\PYZat{@}
\def\PYZlb{[}
\def\PYZrb{]}
\makeatother


    % Exact colors from NB
    \definecolor{incolor}{rgb}{0.0, 0.0, 0.5}
    \definecolor{outcolor}{rgb}{0.545, 0.0, 0.0}



    
    % Prevent overflowing lines due to hard-to-break entities
    \sloppy 
    % Setup hyperref package
    \hypersetup{
      breaklinks=true,  % so long urls are correctly broken across lines
      colorlinks=true,
      urlcolor=urlcolor,
      linkcolor=linkcolor,
      citecolor=citecolor,
      }
    % Slightly bigger margins than the latex defaults
    
    \geometry{verbose,tmargin=1in,bmargin=1in,lmargin=1in,rmargin=1in}
    
    

    \begin{document}
    \title{Practica 1}
    \author{González Borja, Miguel \\ Illescas Arizti, Rodrigo \\ Meyer Mañón, Juan Carlos\\ Rodríguez Orozco, Alejandro}
    \date{11 / Marzo / 2018}

    \maketitle
    
    \section*{Introducci\'on}
A continuación los ejercicios de la práctica. Todos fueron realizados en Python 3.6 utilizando los paquetes \textit{numpy} y \textit{scipy.linalg}. También se incluye un cuaderno de IPython para cada ejercicio dentro de la carpeta Ejercicios, junto con los scripts \textit{.py}. Cada uno de estos utiliza los siguientes imports:
    
    \begin{Verbatim}[commandchars=\\\{\}]
{\color{incolor}In [{\color{incolor}1}]:} \PY{k+kn}{import} \PY{n+nn}{sys}
        \PY{n}{sys}\PY{o}{.}\PY{n}{path}\PY{o}{.}\PY{n}{append}\PY{p}{(}\PY{l+s+s1}{\PYZsq{}}\PY{l+s+s1}{../}\PY{l+s+s1}{\PYZsq{}}\PY{p}{)}
        \PY{k+kn}{from} \PY{n+nn}{IPython}\PY{n+nn}{.}\PY{n+nn}{display} \PY{k}{import} \PY{n}{Latex}
        \PY{k+kn}{import} \PY{n+nn}{latexStrings} \PY{k}{as} \PY{n+nn}{ls}
        \PY{k+kn}{import} \PY{n+nn}{numpy} \PY{k}{as} \PY{n+nn}{np}
        \PY{k+kn}{import} \PY{n+nn}{scipy}\PY{n+nn}{.}\PY{n+nn}{linalg} \PY{k}{as} \PY{n+nn}{linear}
        \PY{k+kn}{import} \PY{n+nn}{eigenvalues} \PY{k}{as} \PY{n+nn}{ev}
\end{Verbatim}
Se puede encontrar un repositorio del proyecto \href{https://github.com/Alex23rodriguez/CalcNumerico.git}{aqui}.

\section{Ejercicio 1}\label{ejercicio-1}

Tenemos la matriz \(A\) y el vector \(q_0\) definidos como:

    \begin{Verbatim}[commandchars=\\\{\}]
{\color{incolor}In [{\color{incolor}1}]:} \PY{n}{A} \PY{o}{=} \PY{n}{np}\PY{o}{.}\PY{n}{array}\PY{p}{(}\PY{p}{[}\PY{p}{[}\PY{l+m+mi}{1}\PY{p}{,}\PY{l+m+mi}{1}\PY{p}{,}\PY{l+m+mi}{2}\PY{p}{]}\PY{p}{,}\PY{p}{[}\PY{o}{\PYZhy{}}\PY{l+m+mi}{1}\PY{p}{,}\PY{l+m+mi}{9}\PY{p}{,}\PY{l+m+mi}{3}\PY{p}{]}\PY{p}{,}\PY{p}{[}\PY{l+m+mi}{0}\PY{p}{,}\PY{o}{\PYZhy{}}\PY{l+m+mi}{1}\PY{p}{,}\PY{l+m+mi}{3}\PY{p}{]}\PY{p}{]}\PY{p}{)}
        \PY{n}{q} \PY{o}{=} \PY{n}{np}\PY{o}{.}\PY{n}{array}\PY{p}{(}\PY{p}{[}\PY{l+m+mi}{1}\PY{p}{,}\PY{l+m+mi}{1}\PY{p}{,}\PY{l+m+mi}{1}\PY{p}{]}\PY{p}{)}
        
\end{Verbatim}
\[ 
A = 
\begin{pmatrix} 
1 & 1 & 2 \\ 
-1 & 9 & 3 \\ 
0 & -1 & 3 \\ 
\end{pmatrix}
 \qquad
 \vec{q_0} = 
\begin{pmatrix} 
1 \\ 
1 \\ 
1 \\ 
\end{pmatrix}
 \]

Queremos calcular 10 iteraciones del metodo de la potencia. Esto nos da
como resultado:

\begin{Verbatim}[commandchars=\\\{\}]
{\color{incolor}In [{\color{incolor}2}]:} \PY{p}{[}\PY{n}{q10}\PY{p}{,} \PY{n}{l10} \PY{p}{,}\PY{n}{iterations}\PY{p}{]}\PY{o}{=}\PY{n}{ev}\PY{o}{.}\PY{n}{powerMethod}\PY{p}{(}\PY{n}{A}\PY{p}{,}\PY{n}{q}\PY{p}{,}\PY{l+m+mf}{1e\PYZhy{}6}\PY{p}{,}\PY{l+m+mi}{10}\PY{p}{)}
\end{Verbatim}    
    
Numero de Iteraciones: 10
\[ 
 \vec{q_{10}} = 
\begin{pmatrix} 
0.083519 \\ 
0.979588 \\ 
-0.182845 \\ 
\end{pmatrix}
\qquad \lambda = 8.35525106702442
 \]

    

Despues comparemos los resultados con los valores 'exactos' calculados por
el paquete \textit{scipy.linalg}:

\begin{Verbatim}[commandchars=\\\{\}]
{\color{incolor}In [{\color{incolor}3}]:} \PY{p}{[}\PY{n}{L}\PY{p}{,}\PY{n}{V}\PY{p}{]} \PY{o}{=} \PY{n}{linear}\PY{o}{.}\PY{n}{eig}\PY{p}{(}\PY{n}{A}\PY{p}{)}
\end{Verbatim}
 \[ 
 \vec{\lambda} = 
\begin{pmatrix} 
8.354545 \\ 
1.224672 \\ 
3.420784 \\ 
\end{pmatrix}
 \qquad
V = 
\begin{pmatrix} 
0.083444 & -0.992728 & 0.515311 \\ 
0.979576 & -0.104882 & -0.332386 \\ 
-0.182943 & -0.059078 & 0.789921 \\ 
\end{pmatrix}
 \]

Vemos que en efecto el método de la potencia calculo el eigenvector
dominante de la matriz. Esto se debe a que se cumplen las condiciones
del método, es decir: 
\begin{enumerate}
\item A tiene un eigenvalor dominante (8.3545) 
\item Nuestro vector inicial $q_0$ puede ser escrito como combinación lineal
de los eigenvectores de A con coeficientes todos distintos de 0
\end{enumerate}

Ahora, tomemos el eigenvector asociado al eigenvalos dominante, dado
por:

\begin{Verbatim}[commandchars=\\\{\}]
{\color{incolor}In [{\color{incolor}4}]:} \PY{n}{v}\PY{o}{=}\PY{n}{V}\PY{p}{[}\PY{p}{:}\PY{p}{,}\PY{l+m+mi}{0}\PY{p}{]}
\end{Verbatim}    
 \[ 
 \vec{v} = 
\begin{pmatrix} 
0.083444 \\ 
0.979576 \\ 
-0.182943 \\ 
\end{pmatrix}
 \]

Y con esto queremos calcular las razones de convergencia en cada paso de la
iteración, dadas por:
\[ \widetilde{r} = \{r_i\}, \qquad r_i = \frac{\Vert q_{i}-v \Vert_\infty}{\Vert q_{i-1}-v \Vert_\infty}, \qquad i\in \{1,2,...,10\} \]

\begin{Verbatim}[commandchars=\\\{\}]
{\color{incolor}In [{\color{incolor}5}]:} \PY{n}{ratios}\PY{o}{=}\PY{p}{[}\PY{p}{]}
        \PY{n}{prevq}\PY{o}{=}\PY{n}{q}
        \PY{k}{for} \PY{n}{i} \PY{o+ow}{in} \PY{n+nb}{range}\PY{p}{(}\PY{l+m+mi}{1}\PY{p}{,}\PY{l+m+mi}{11}\PY{p}{)}\PY{p}{:}
            \PY{p}{[}\PY{n}{currentq}\PY{p}{,}\PY{n}{\PYZus{}}\PY{p}{,}\PY{n}{\PYZus{}}\PY{p}{]} \PY{o}{=} \PY{n}{ev}\PY{o}{.}\PY{n}{powerMethod}\PY{p}{(}\PY{n}{A}\PY{p}{,}\PY{n}{q}\PY{p}{,}\PY{l+m+mf}{1e\PYZhy{}6}\PY{p}{,}\PY{n}{i}\PY{p}{)}
            \PY{n}{ratio} \PY{o}{=} \PY{n}{linear}\PY{o}{.}\PY{n}{norm}\PY{p}{(}\PY{n}{currentq}\PY{o}{\PYZhy{}}\PY{n}{v}\PY{p}{,}\PY{n}{np}\PY{o}{.}\PY{n}{inf}\PY{p}{)}\PY{o}{/}\PY{n}{linear}\PY{o}{.}\PY{n}{norm}\PY{p}{(}\PY{n}{prevq}\PY{o}{\PYZhy{}}\PY{n}{v}\PY{p}{,}\PY{n}{np}\PY{o}{.}\PY{n}{inf}\PY{p}{)}
            \PY{n}{ratios}\PY{o}{.}\PY{n}{append}\PY{p}{(}\PY{n}{ratio}\PY{p}{)}
            \PY{n}{prevq} \PY{o}{=} \PY{n}{currentq}
\end{Verbatim}

\[ 
 \widetilde{r} = 
\{ 
0.297033, 0.382353, 0.390998, 0.400885, 0.405792, 0.407929, 0.408825, 0.409194, 0.409346, 0.409409\}
 \]

Observemos que en efecto, las razones de cada iteracion rapidamente al
valor teorico, dado por:
\[ r = \left| \frac{\lambda_2}{\lambda_1} \right| = 0.40945181373495726 \]


\section{Ejercicio 2.1}\label{ejercicio-2.1}
Utilizando la matriz \(A\) y el vector \(q_0\) del ejercicio anterior
definidos como sigue:
\begin{Verbatim}[commandchars=\\\{\}]
{\color{incolor}In [{\color{incolor}1}]:} \PY{n}{A} \PY{o}{=} \PY{n}{np}\PY{o}{.}\PY{n}{array}\PY{p}{(}\PY{p}{[}\PY{p}{[}\PY{l+m+mi}{1}\PY{p}{,}\PY{l+m+mi}{1}\PY{p}{,}\PY{l+m+mi}{2}\PY{p}{]}\PY{p}{,}\PY{p}{[}\PY{o}{\PYZhy{}}\PY{l+m+mi}{1}\PY{p}{,}\PY{l+m+mi}{9}\PY{p}{,}\PY{l+m+mi}{3}\PY{p}{]}\PY{p}{,}\PY{p}{[}\PY{l+m+mi}{0}\PY{p}{,}\PY{o}{\PYZhy{}}\PY{l+m+mi}{1}\PY{p}{,}\PY{l+m+mi}{3}\PY{p}{]}\PY{p}{]}\PY{p}{)}
        \PY{n}{q} \PY{o}{=} \PY{n}{np}\PY{o}{.}\PY{n}{array}\PY{p}{(}\PY{p}{[}\PY{l+m+mi}{1}\PY{p}{,}\PY{l+m+mi}{1}\PY{p}{,}\PY{l+m+mi}{1}\PY{p}{]}\PY{p}{)}
\end{Verbatim}

\[ 
A = 
\begin{pmatrix} 
1 & 1 & 2 \\ 
-1 & 9 & 3 \\ 
0 & -1 & 3 \\ 
\end{pmatrix}
 \quad
 \vec{q_0} = 
\begin{pmatrix} 
1 \\ 
1 \\ 
1 \\ 
\end{pmatrix}
 \]

Se calcularán 10 iteraciones del método de la potencia con shift
\(\rho_1\) y \(\rho_2\) y donde \(\vec{q}_{10}\) es el vector
que se aporxima al eigenvector y \(\sigma_{10}\) el eigenvalor
aproximado después de 10 iteraciones.
Para \(\rho_1=0\), que es simplemente aplicar el método de la potencia
inversa, se obtiene:

\begin{Verbatim}[commandchars=\\\{\}]
{\color{incolor}In [{\color{incolor}2}]:} \PY{p}{[}\PY{n}{q10}\PY{p}{,} \PY{n}{l10} \PY{p}{,}\PY{n}{iterations}\PY{p}{]}\PY{o}{=}\PY{n}{ev}\PY{o}{.}\PY{n}{inversePowerShift}\PY{p}{(}\PY{n}{A}\PY{p}{,}\PY{n}{q}\PY{p}{,}\PY{l+m+mi}{0}\PY{p}{,}\PY{l+m+mf}{1e\PYZhy{}6}\PY{p}{,}\PY{l+m+mi}{10}\PY{p}{)}
\end{Verbatim}

Numero de Iteraciones: 10\[ 
\vec{q_{10}} = 
\begin{pmatrix} 
0.992719 \\ 
0.104174 \\ 
0.060466 \\ 
\end{pmatrix}
 \quad \sigma_{10} = 1.2267894261411831\]

Comparemos los resultados anteriores con los valores "exactos"
calculados por el paquete \emph{scipy.linalg}:
\begin{Verbatim}[commandchars=\\\{\}]
{\color{incolor}In [{\color{incolor}3}]:} \PY{p}{[}\PY{n}{L}\PY{p}{,}\PY{n}{V}\PY{p}{]} \PY{o}{=} \PY{n}{linear}\PY{o}{.}\PY{n}{eig}\PY{p}{(}\PY{n}{A}\PY{p}{)}
        \PY{p}{[}\PY{n}{L}\PY{p}{,}\PY{n}{V}\PY{p}{]} \PY{o}{=} \PY{n}{ev}\PY{o}{.}\PY{n}{pairSort}\PY{p}{(}\PY{n}{L}\PY{p}{,}\PY{n}{V}\PY{p}{)}
\end{Verbatim}
    
\[ 
\vec{\lambda} = 
\begin{pmatrix} 
8.354545 \\ 
3.420784 \\ 
1.224672 \\ 
\end{pmatrix}
 \quad
V = 
\begin{pmatrix} 
0.083444 & 0.515311 & -0.992728 \\ 
0.979576 & -0.332386 & -0.104882 \\ 
-0.182943 & 0.789921 & -0.059078 \\ 
\end{pmatrix}
 \]

\(\vec{\lambda}\) es el vector que contiene a los tres
eigenvalores "exactos" de la matriz \(A\) ordenados por magnitud y las
columnas de la matriz \(V\) son los eigenvectores "exactos" de \(A\).

Se observa que \(\sigma_{10}=1.22678942614\) tiene dos decimales iguales
a \(\lambda_{3}=1.224672\) , la tercera entrada de
\(\vec{\lambda}\). El método de la potencia inversa converge
teóricamente al menor eigenpar de \(A\) bajo las siguientes condiciones:
\begin{enumerate}
\item \(A\) tiene un menor eigenvalor, es decir
\(|\lambda_1| > |\lambda_2| > |\lambda_3|\) (En este caso 1.2246) 
\item \(\vec{q}_0\) puede ser escrito como combinacion lineal de los
eigenvectores de A con los coeficientes de \(\vec{v}_2\) y \(\vec{v}_3\)
distintos de 0. En este caso, \(\vec{q}_0\) claramente no es ortogonal a
los vectores de la matriz \(V\), por lo tanto ninguno de los
coeficientes de la combinacion lineal sera 0.
\end{enumerate}
Ahora comparemos \(\vec{v}_3\), el eigenvector asignado a \(\lambda_3\),
con \(\vec{q}_{10}\).
\begin{Verbatim}[commandchars=\\\{\}]
{\color{incolor}In [{\color{incolor}4}]:} \PY{n}{v}\PY{o}{=}\PY{n}{V}\PY{p}{[}\PY{p}{:}\PY{p}{,}\PY{l+m+mi}{2}\PY{p}{]}
\end{Verbatim}

\[ 
 \vec{v_3} = 
\begin{pmatrix} 
-0.992728 \\ 
-0.104882 \\ 
-0.059078 \\ 
\end{pmatrix}
 \quad
 \vec{q_{10}} = 
\begin{pmatrix} 
0.992719 \\ 
0.104174 \\ 
0.060466 \\ 
\end{pmatrix}
 \]

Observamos que \(\vec{q}_{10}\) es apximadamente $
-\vec{v_3}$. Esto se debe a que el método de la potencia
inversa puede converger a \(\pm\vec{v_3}\). Entonces de ahora
en adelante nos referiremos a \(\vec{v_3}\) por
\(-\vec{v_3}\) :

\begin{Verbatim}[commandchars=\\\{\}]
{\color{incolor}In [{\color{incolor}5}]:} \PY{n}{v}\PY{o}{=}\PY{o}{\PYZhy{}}\PY{n}{v}
\end{Verbatim}

\texttt{\color{outcolor}Out[{\color{outcolor}6}]:}
    
\[ 
\vec{v_3} = 
\begin{pmatrix} 
0.992728 \\ 
0.104882 \\ 
0.059078 \\ 
\end{pmatrix}
 \]

Calculamos las razones de convergencia en cada paso de la iteración,
dadas por:
\[ \widetilde{r} = \{r_i\}, \qquad r_i = \frac{\Vert q_{i}-v \Vert_\infty}{\Vert q_{i-1}-v \Vert_\infty}, \qquad i\in \{1,2,...,10\} \]
\begin{Verbatim}[commandchars=\\\{\}]
{\color{incolor}In [{\color{incolor}6}]:} \PY{n}{ratios}\PY{o}{=}\PY{p}{[}\PY{p}{]}
        \PY{n}{prevq}\PY{o}{=}\PY{n}{q}
        \PY{k}{for} \PY{n}{i} \PY{o+ow}{in} \PY{n+nb}{range}\PY{p}{(}\PY{l+m+mi}{1}\PY{p}{,}\PY{l+m+mi}{10}\PY{p}{)}\PY{p}{:}
            \PY{p}{[}\PY{n}{currentq}\PY{p}{,}\PY{n}{\PYZus{}}\PY{p}{,}\PY{n}{\PYZus{}}\PY{p}{]} \PY{o}{=} \PY{n}{ev}\PY{o}{.}\PY{n}{inversePowerShift}\PY{p}{(}\PY{n}{A}\PY{p}{,}\PY{n}{q}\PY{p}{,}\PY{l+m+mi}{0}\PY{p}{,}\PY{l+m+mf}{1e\PYZhy{}6}\PY{p}{,}\PY{n}{i}\PY{p}{)}
            \PY{n}{ratio} \PY{o}{=} \PY{n}{linear}\PY{o}{.}\PY{n}{norm}\PY{p}{(}\PY{n}{currentq}\PY{o}{\PYZhy{}}\PY{n}{v}\PY{p}{,}\PY{n}{np}\PY{o}{.}\PY{n}{inf}\PY{p}{)}\PY{o}{/}\PY{n}{linear}\PY{o}{.}\PY{n}{norm}\PY{p}{(}\PY{n}{prevq}\PY{o}{\PYZhy{}}\PY{n}{v}\PY{p}{,}\PY{n}{np}\PY{o}{.}\PY{n}{inf}\PY{p}{)}
            \PY{n}{ratios}\PY{o}{.}\PY{n}{append}\PY{p}{(}\PY{n}{ratio}\PY{p}{)}
            \PY{n}{prevq} \PY{o}{=} \PY{n}{currentq}
\end{Verbatim}

\[ 
\widetilde{r} = 
\{ 
0.752001, 0.966084, 0.853759, 0.678343, 0.495284, 0.404372, 0.373393, 0.363308, 0.359880\}
\]

Ahora veamos la razón de convergencia teórica, dada por:
\[ r = \left| \frac{\lambda_3}{\lambda_2} \right| = 0.35800909831776 \]
Se oberva que, en efecto, \(r_{10}\) = \(0.359880\) es aproximademente
la razón teórica \(r=0.35800909831776\)

\section{Ejercicio 2.2}\label{ejercicio-2.2}
Para la misma matriz \(A\) y vector \(q_0\) defininidos anteriormente,
se realizarán las mismas pruebas para un shift \(\rho_2=3.3\). Se
encontrarán nuevos: vector \(\vec{q}_{10}\), la aproximación
de eigenvector, y \(\sigma_{10}\) la aproximación del eigenvalor.

\begin{Verbatim}[commandchars=\\\{\}]
{\color{incolor}In [{\color{incolor}1}]:} \PY{p}{[}\PY{n}{q10}\PY{p}{,} \PY{n}{l10} \PY{p}{,}\PY{n}{iterations}\PY{p}{]}\PY{o}{=}\PY{n}{ev}\PY{o}{.}\PY{n}{inversePowerShift}\PY{p}{(}\PY{n}{A}\PY{p}{,}\PY{n}{q}\PY{p}{,}\PY{l+m+mf}{3.3}\PY{p}{,}\PY{l+m+mf}{1e\PYZhy{}6}\PY{p}{,}\PY{l+m+mi}{10}\PY{p}{)}
\end{Verbatim}

    
Numero de Iteraciones: 4
\[ 
 \vec{q_{10}} = 
\begin{pmatrix} 
0.515311 \\ 
-0.332385 \\ 
0.789921 \\ 
\end{pmatrix}
 \quad \sigma_{10} = 3.420782623785237 
 \]

La primera observación es que el proceso sólo hizo 4 iteraciones, esto
quiere decir que el método alcanzó el criterio de error relativo sin
necesidad de hacer las 10 iteraciones. Comparemos los resultados
anteriores con los valores "exactos" calculados por el paquete
\emph{scipy.lingalg} calculados anteriormente
  
\[ 
 \vec{\lambda} = 
\begin{pmatrix} 
8.354545 \\ 
3.420784 \\ 
1.224672 \\ 
\end{pmatrix}
 \quad 
V = 
\begin{pmatrix} 
0.083444 & 0.515311 & -0.992728 \\ 
0.979576 & -0.332386 & -0.104882 \\ 
-0.182943 & 0.789921 & -0.059078 \\ 
\end{pmatrix}
 \]

    

El vector \(\vec{q}_{10}\) es aproximadamente al eigenvector
"exacto" \(\vec{v_{2}}\), segunda columna de la matriz \(V\).
Ademas, el valor \(\sigma_{10}\) tiene 5 cifras decimales iguales al
eigenvalor "exacto" \(\lambda_2\).

Observamos lo siguiente:
\[|\lambda_2-\rho_2|<|\lambda_3-\rho_2|<|\lambda_1-\rho_2| \Rightarrow |\lambda_2-\rho_2|^{-1}>|\lambda_3-\rho_2|^{-1}>|\lambda_1-\rho_2|^{-1}\]

Con un razonamiento similar al ejercicio anterior, al tener la
desigualdad anterior, el método de la potencia inversa con shift
\(\rho_2=3.3\) cumple una de las hipótesis y el converge al valor
\((\lambda_2-\rho_2)^{-1}\) y el eigenvector asignado
\(\vec{v}_2\). Luego con un simple despeje obtenemos
\(\lambda_2\) y con ello el eigenpar de \(A\)
\((\lambda_2,\vec{v}_2)\).

Nuevamente, el vector inicial \(\vec{q_{0}}\) no es ortogonal
a ninguno de los vectores de la matriz \(V\), por lo tanto
\(\vec{q_{0}}\) se puede escribir como combinanción lineal de
los vectores de \(V\) con coeficientes todos distintos de 0, cumpliendo
la segunda hipótesis del metodo, justificando asi la convergencia.

Calculamos las razones de convergencia en cada paso de la iteración, dadas por:
\[ \widetilde{r} = \{r_i\}, \qquad r_i = \frac{\Vert q_{i}-v \Vert_\infty}{\Vert q_{i-1}-v \Vert_\infty}, \qquad i\in \{1,2,...,10\} \]
\begin{Verbatim}[commandchars=\\\{\}]
{\color{incolor}In [{\color{incolor}2}]:} \PY{n}{ratios}\PY{o}{=}\PY{p}{[}\PY{p}{]}
         \PY{n}{v}\PY{o}{=}\PY{n}{V}\PY{p}{[}\PY{p}{:}\PY{p}{,}\PY{l+m+mi}{1}\PY{p}{]}
         \PY{n}{prevq}\PY{o}{=}\PY{n}{q}
         \PY{k}{for} \PY{n}{i} \PY{o+ow}{in} \PY{n+nb}{range}\PY{p}{(}\PY{l+m+mi}{1}\PY{p}{,}\PY{l+m+mi}{11}\PY{p}{)}\PY{p}{:}
             \PY{p}{[}\PY{n}{currentq}\PY{p}{,}\PY{n}{\PYZus{}}\PY{p}{,}\PY{n}{\PYZus{}}\PY{p}{]} \PY{o}{=} \PY{n}{ev}\PY{o}{.}\PY{n}{inversePowerShift}\PY{p}{(}\PY{n}{A}\PY{p}{,}\PY{n}{q}\PY{p}{,}\PY{l+m+mf}{3.3}\PY{p}{,}\PY{l+m+mi}{0}\PY{p}{,}\PY{n}{i}\PY{p}{)}
             \PY{n}{ratio} \PY{o}{=} \PY{n}{linear}\PY{o}{.}\PY{n}{norm}\PY{p}{(}\PY{n}{currentq}\PY{o}{\PYZhy{}}\PY{n}{v}\PY{p}{,}\PY{n}{np}\PY{o}{.}\PY{n}{inf}\PY{p}{)}\PY{o}{/}\PY{n}{linear}\PY{o}{.}\PY{n}{norm}\PY{p}{(}\PY{n}{prevq}\PY{o}{\PYZhy{}}\PY{n}{v}\PY{p}{,}\PY{n}{np}\PY{o}{.}\PY{n}{inf}\PY{p}{)}
             \PY{n}{ratios}\PY{o}{.}\PY{n}{append}\PY{p}{(}\PY{n}{ratio}\PY{p}{)}
             \PY{n}{prevq} \PY{o}{=} \PY{n}{currentq}
\end{Verbatim}

\begin{align*}
 \widetilde{r} = 
\{ & 
0.014461, 0.024879, 0.020834, 0.032431, 0.021112, \\
& 0.085590, 0.050552, 0.061824, 0.056628, 0.062165\}
\end{align*}
 
Ahora veamos la razón de convergencia teórica, dada por:
\[ r = \left| \frac{\lambda_2-\rho_2}{\lambda_3-\rho_2} \right| = 0.05819972269618957 \]
Observamos que las razones \(r_i\) comienzan a oscilar alrededor de
\(r=0.05819972269618957\) (razón teórica) desde \(i=7\). Esto se debe a
que como el método converge rápidamente y con más iteraciones la
aproximación se vuelve prácticamente igual al eigenpar busacado. De
hecho, si aumentamos las razones calculadas, veremos que:

    \begin{Verbatim}[commandchars=\\\{\}]
{\color{incolor}In [{\color{incolor}3}]:} \PY{k}{for} \PY{n}{i} \PY{o+ow}{in} \PY{n+nb}{range}\PY{p}{(}\PY{l+m+mi}{11}\PY{p}{,}\PY{l+m+mi}{21}\PY{p}{)}\PY{p}{:}
             \PY{p}{[}\PY{n}{currentq}\PY{p}{,}\PY{n}{\PYZus{}}\PY{p}{,}\PY{n}{\PYZus{}}\PY{p}{]} \PY{o}{=} \PY{n}{ev}\PY{o}{.}\PY{n}{inversePowerShift}\PY{p}{(}\PY{n}{A}\PY{p}{,}\PY{n}{q}\PY{p}{,}\PY{l+m+mf}{3.3}\PY{p}{,}\PY{l+m+mi}{0}\PY{p}{,}\PY{n}{i}\PY{p}{)}
             \PY{n}{ratio} \PY{o}{=} \PY{n}{linear}\PY{o}{.}\PY{n}{norm}\PY{p}{(}\PY{n}{currentq}\PY{o}{\PYZhy{}}\PY{n}{v}\PY{p}{,}\PY{n}{np}\PY{o}{.}\PY{n}{inf}\PY{p}{)}\PY{o}{/}\PY{n}{linear}\PY{o}{.}\PY{n}{norm}\PY{p}{(}\PY{n}{prevq}\PY{o}{\PYZhy{}}\PY{n}{v}\PY{p}{,}\PY{n}{np}\PY{o}{.}\PY{n}{inf}\PY{p}{)}
             \PY{n}{ratios}\PY{o}{.}\PY{n}{append}\PY{p}{(}\PY{n}{ratio}\PY{p}{)}
             \PY{n}{prevq} \PY{o}{=} \PY{n}{currentq}
\end{Verbatim}
    
\begin{align*}
\widetilde{r}_{10-20} =
\{ & 0.062165, 0.172414, 1.100000, 1.045455, 0.913043, 1.095238, \\
&  0.913043, 1.095238, 0.913043, 1.095238, 0.913043\}
\end{align*}
La razones de la práctica empiezan a acercarse a 1. Esto se debe a que
\(\vec{q}_{i}-\vec{v}\) es casi igual a \(\vec{q}_{i-1}-\vec{v}\) por lo
que la razón se acerca a 1.

\section{Ejercicio 3}\label{ejercicio-3}
Tomando \(A\) y \(q_0\) como los definimos anteriormente:

\begin{Verbatim}[commandchars=\\\{\}]
{\color{incolor}In [{\color{incolor}1}]:} \PY{n}{A} \PY{o}{=} \PY{n}{np}\PY{o}{.}\PY{n}{array}\PY{p}{(}\PY{p}{[}\PY{p}{[}\PY{l+m+mi}{1}\PY{p}{,}\PY{l+m+mi}{1}\PY{p}{,}\PY{l+m+mi}{2}\PY{p}{]}\PY{p}{,}\PY{p}{[}\PY{o}{\PYZhy{}}\PY{l+m+mi}{1}\PY{p}{,}\PY{l+m+mi}{9}\PY{p}{,}\PY{l+m+mi}{3}\PY{p}{]}\PY{p}{,}\PY{p}{[}\PY{l+m+mi}{0}\PY{p}{,}\PY{o}{\PYZhy{}}\PY{l+m+mi}{1}\PY{p}{,}\PY{l+m+mi}{3}\PY{p}{]}\PY{p}{]}\PY{p}{)}
        \PY{n}{q} \PY{o}{=} \PY{n}{np}\PY{o}{.}\PY{n}{array}\PY{p}{(}\PY{p}{[}\PY{l+m+mi}{1}\PY{p}{,}\PY{l+m+mi}{1}\PY{p}{,}\PY{l+m+mi}{1}\PY{p}{]}\PY{p}{)}
\end{Verbatim}

 \[ 
A = 
\begin{pmatrix} 
1 & 1 & 2 \\ 
-1 & 9 & 3 \\ 
0 & -1 & 3 \\ 
\end{pmatrix}
  \quad
\vec{q_0} = 
\begin{pmatrix} 
1 \\ 
1 \\ 
1 \\ 
\end{pmatrix}
 \]

    

Se calcularán 10 iteraciones del método de la potencia con shift
\(\rho=3.6\) , donde \(q_{10}\) es el vector que se aporxima al
eigenvector y \(\sigma_{10}\) el eigenvalor aproximado después de 10
iteraciones
\begin{Verbatim}[commandchars=\\\{\}]
{\color{incolor}In [{\color{incolor}2}]:} \PY{p}{[}\PY{n}{q10}\PY{p}{,} \PY{n}{l10} \PY{p}{,}\PY{n}{iterations}\PY{p}{]}\PY{o}{=}\PY{n}{ev}\PY{o}{.}\PY{n}{inversePowerShift}\PY{p}{(}\PY{n}{A}\PY{p}{,}\PY{n}{q}\PY{p}{,}\PY{l+m+mf}{3.6}\PY{p}{,}\PY{l+m+mf}{1e\PYZhy{}15}\PY{p}{)}
\end{Verbatim}
    
 Numero de Iteraciones: 12\[ 
 \vec{q_{10}} = 
\begin{pmatrix} 
0.5153107267324357 \\ 
-0.33238561122815846 \\ 
0.7899206671324485 \\ 
\end{pmatrix}
 \quad
 \sigma_{10} = 3.4207835356869145 
 \]   

El método requiere 12 iteraciones para cumplir con el criterio de error relativo igual a \(10^{-15}\), es decir
\[\frac{\Vert A\vec{q}_{10} - \sigma_{10}\vec{q}_{10}\Vert}{\Vert A\vec{q}_{10}\Vert} \leq 10^{-15}\]

Comparemos esto con los valores exactos, dados por:
\begin{Verbatim}[commandchars=\\\{\}]
{\color{incolor}In [{\color{incolor}3}]:} \PY{p}{[}\PY{n}{L}\PY{p}{,}\PY{n}{V}\PY{p}{]} \PY{o}{=} \PY{n}{linear}\PY{o}{.}\PY{n}{eig}\PY{p}{(}\PY{n}{A}\PY{p}{)}
\end{Verbatim}
   
\[ 
 \vec{v} = 
\begin{pmatrix} 
0.515310726732435 \\ 
-0.3323856112281598 \\ 
0.7899206671324484 \\ 
\end{pmatrix}
 \quad 
 \lambda = 3.420783535686916 
 \]

    

En efecto, vemos que los valores tienen alrededor de 15 cifras correctas
con respecto a los calculados por \emph{scipy}

¿Por qué converge tan rápido el metodo? Veamos todos los eigenvalores de
la matriz con el shift \(\rho=3.6\):

 \begin{Verbatim}[commandchars=\\\{\}]
{\color{incolor}In [{\color{incolor}4}]:} \PY{p}{[}\PY{n}{L}\PY{p}{,}\PY{n}{V}\PY{p}{]} \PY{o}{=} \PY{n}{linear}\PY{o}{.}\PY{n}{eig}\PY{p}{(}\PY{n}{A}\PY{o}{\PYZhy{}}\PY{l+m+mf}{3.6}\PY{o}{*}\PY{n}{np}\PY{o}{.}\PY{n}{identity}\PY{p}{(}\PY{n+nb}{len}\PY{p}{(}\PY{n}{A}\PY{p}{)}\PY{p}{)}\PY{p}{)}
\end{Verbatim}

\[ 
 \vec{\lambda} = 
\begin{pmatrix} 
4.754545 \\ 
-2.375328 \\ 
-0.179216 \\ 
\end{pmatrix}
 \quad 
V = 
\begin{pmatrix} 
0.083444 & -0.992728 & 0.515311 \\ 
0.979576 & -0.104882 & -0.332386 \\ 
-0.182943 & -0.059078 & 0.789921 \\ 
\end{pmatrix}
 \]

El método converge "rápido" porque
\(|\lambda_3-\rho_3|=|\lambda_3-3.6|=0.179216464313\) es cercano a 0 y
la razón de convergencia teórica \(r\) es tal que:
\[r=\left| \frac{\lambda_3 - \rho_3}{\lambda_2 - \rho_2}\right |=0.07544913221790368\]


    % Add a bibliography block to the postdoc
    
    
    
    \end{document}
